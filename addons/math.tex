%!TEX root=../../main.tex

%-------------------------------------------------------------------------------
% Packages
%-------------------------------------------------------------------------------

\usepackage{amsmath} 
% \usepackage{amsthm} % proof command conflict
\usepackage{amssymb}
\usepackage{mathrsfs} % Some math fonts
\usepackage{mathtools}
\usepackage{xspace}
\usepackage{stmaryrd} % for integer brackets

%-------------------------------------------------------------------------------
% Form and style
% \renewcommand at the beginning to change style for whole document
% (the space at the begining of the lines is on purpose)
%-------------------------------------------------------------------------------

 \newcommand{\styleVec}[1]{\ensuremath{\overrightarrow{#1}}\xspace}
 \newcommand{\styleBBSet}[1]{\ensuremath{\mathbb{#1}}\xspace}
 \newcommand{\styleGroup}[1]{\ensuremath{\mathbb{#1}}\xspace}
 \newcommand{\styleRing}[1]{\ensuremath{\mathcal{#1}}\xspace}
 \newcommand{\styleRelation}[1]{\ensuremath{\mathcal{#1}}\xspace}
 \newcommand{\styleEquation}[1]{\ensuremath{\mathcal{#1}}\xspace}
 \newcommand{\styleProba}[3]{
	{\Pr}_{#3}\ifthenelse{\equal{#2}{}}{\brackets{#1}}{\bracketsSuchThat{#1}{#2}}}

%-------------------------------------------------------------------------------
% Macros
%-------------------------------------------------------------------------------

\renewcommand{\vec}[1]{\styleVec{#1}}

% N, Z, etc
\newcommand{\N}{\styleBBSet{N}}
\newcommand{\Z}{\styleBBSet{Z}}
\newcommand{\Zn}{\Z_n}
\newcommand{\Zp}{\Z_p}
\newcommand{\Q}{\styleBBSet{Q}}
\newcommand{\R}{\styleBBSet{R}}
\newcommand{\C}{\styleBBSet{C}}
\newcommand{\F}{\styleBBSet{F}}

% Groups
\newcommand{\GroupG}{\styleGroup{G}}
\newcommand{\Group}[1]{\styleGroup{#1}}

% Rings
\newcommand{\RingR}{\styleRing{R}}
\newcommand{\Ring}[1]{\styleRing{#1}}

% Relations
\newcommand{\RelR}{\styleRelation{R}}
\newcommand{\Rel}[1]{\styleRelation{#1}}

% Equations
\newcommand{\EquationE}{\styleEquation{E}}
\newcommand{\EqE}{\EquationE}
\newcommand{\Equation}[1]{\styleEquation{#1}}

% Parentheses, brackets, etc
\newcommand{\parentheses}[1]{\left( #1 \right)}
\newcommand{\set}[1]{\left\{ #1 \right\}}
\newcommand{\integerSet}[1]{\llbracket #1 \rrbracket}
\newcommand{\setSuchThat}[2]{\set{#1 \; \middle| \; #2}}
\newcommand{\brackets}[1]{\left[ #1 \right]}
\newcommand{\bracketsSuchThat}[2]{\brackets{\: #1 \; \middle| \; #2 \:}}
\newcommand{\scal}[2]{\left< #1, #2 \right> }

% Probabilities
\newcommand{\Proba}[1]{\styleProba{#1}{}{}}
\newcommand{\ProbaCond}[2]{\styleProba{#1}{#2}{}}
\newcommand{\ProbaRand}[2]{\styleProba{#1}{}{#2}}
\newcommand{\ProbaCondRand}[3]{\styleProba{#1}{#2}{#3}}

% Asymptotic
\newcommand{\bigO}[1]{\mathcal{O} \parentheses{#1}}
\newcommand{\smallO}[1]{o\parentheses{#1}}
\newcommand{\bigOmega}[1]{\Omega\parentheses{#1}}
\newcommand{\smallOmega}[1]{\omega\parentheses{#1}}
\newcommand{\bigTheta}[1]{\Theta\parentheses{#1}}

% Operators
\DeclareMathOperator{\Endo}{End}

% Divers
\newcommand{\sigmaperm}{\ensuremath{\mathfrak{S}}\xspace}
