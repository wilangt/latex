\documentclass{beamer}

\usepackage{graphicx} % Allows including images
\usepackage{booktabs} % Allows the use of \toprule, \midrule and \bottomrule in tables

% To put notes in the slides (use with pympress)

\ifnum\speakernotes=1
\usepackage[draft]{pdfcomment} 
\newcommand{\annot}[1]{\marginnote{\pdfcomment[icon=note]{#1}}}
\else
\newcommand{\annot}[1]{}
\fi
\usepackage{environ}
\NewEnviron{speakernote}{
	\annot{\BODY}
}
\newcommand\ret\textCR


\definecolor{rougeb}{rgb}{0.75,0.01,0}

%\setbeamercolor{subsection in toc}{fg=red} %change la couleur de la ToC
\setbeamerfont{section number projected}{family=\rmfamily,series=\bfseries,size=\normalsize}
\setbeamercolor{section number projected}{bg=rougeb,fg=white}
\setbeamerfont{subsection number projected}{family=\rmfamily,series=\bfseries,size=\normalsize}
\setbeamercolor{subsection number projected}{bg=rougeb,fg=white}

\ifnum\tocsection=1
\AtBeginSection[]
{
    \begin{frame}
        \frametitle{Summary}
        \tableofcontents[currentsection]
    \end{frame}
}
\fi

\ifnum\tocsection=2
\AtBeginSubsection[]
{
    \begin{frame}
        \frametitle{Summary}
        \tableofcontents[currentsubsection]
    \end{frame}
}
\fi


%%%%%%%%%%%%%%%%%%%%%%%%%%%%%%%%%%%%%%%%%
% Beamer Presentation
% LaTeX Template
% Version 1.0 (10/11/12)
%
% This template has been downloaded from:
% http://www.LaTeXTemplates.com
%
% License:
% CC BY-NC-SA 3.0 (http://creativecommons.org/licenses/by-nc-sa/3.0/)
%
%%%%%%%%%%%%%%%%%%%%%%%%%%%%%%%%%%%%%%%%%

%----------------------------------------------------------------------------------------
%	PACKAGES AND THEMES
%----------------------------------------------------------------------------------------




\mode<presentation> {

% The Beamer class comes with a number of default slide themes
% which change the colors and layouts of slides. Below this is a list
% of all the themes, uncomment each in turn to see what they look like.

%\usetheme{default}
%\usetheme{AnnArbor}
%\usetheme{Antibes} %pas mal du tout
%\usetheme{Bergen}
%\usetheme{Berkeley}
\usetheme{Berlin} %vraiment excellent -----------------------
%\usetheme{Boadilla}
%\usetheme{CambridgeUS} %bien --> light
%\usetheme{Copenhagen} % bien : ressemble à warsaw
%\usetheme{Darmstadt} %tres tres bien
%\usetheme{Dresden} %excellent (berlin avce la ToC modifiée) --------
%\usetheme{Frankfurt} %bien
%\usetheme{Goettingen}
%\usetheme{Hannover}
%\usetheme{Ilmenau} %excellent (berlin avce la ToC modifiée) --------
%\usetheme{JuanLesPins} % stylé mais peu fonctionnel
%\usetheme{Luebeck} % bien : ressemble à warsaw
%\usetheme{Madrid} % bof
%\usetheme{Malmoe} %\usetheme{Luebeck} % bien : ressemble à warsaw
%\usetheme{Marburg}
%\usetheme{Montpellier} %pas mal du tout ressemble a antibes
%\usetheme{PaloAlto}
%\usetheme{Pittsburgh}
%\usetheme{Rochester}
%\usetheme{Singapore} % berlin en bcp moins bien
%\usetheme{Szeged} %excellent (berlin avce la footer modifié) --------
%\usetheme{Warsaw} %bien

% As well as themes, the Beamer class has a number of color themes
% for any slide theme. Uncomment each of these in turn to see how it
% changes the colors of your current slide theme.

%\usecolortheme{albatross}
\usecolortheme{beaver}
%\usecolortheme{beetle}
%\usecolortheme{crane}
%\usecolortheme{dolphin}
%\usecolortheme{dove}
%\usecolortheme{fly}
%\usecolortheme{lily}
%\usecolortheme{orchid}
%\usecolortheme{rose}
%\usecolortheme{seagull}
%\usecolortheme{seahorse}
%\usecolortheme{whale}
%\usecolortheme{wolverine}

%\setbeamertemplate{footline} % To remove the footer line in all slides uncomment this line
%\setbeamertemplate{footline}[frame number] % To replace the footer line in all slides with a simple slide count uncomment this line

%\setbeamertemplate{navigation symbols}{} % To remove the navigation symbols from the bottom of all slides uncomment this line

\setbeamercovered{transparent} % Fait apparaître les animations en grisé (utile pour la conception, mais peut être commenté lors de la remise du document final)

% Pour utiliser une police à empattements partout
\usefonttheme{serif}

% Pour rajouter la numérotation des frames dans les pieds de page
\newcommand*\oldmacro{}%
\let\oldmacro\insertshorttitle%
\renewcommand*\insertshorttitle{%
  \oldmacro\hfill%
  \insertframenumber\,/\,\inserttotalframenumber}

}
